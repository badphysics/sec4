\documentclass[12pt,letterpaper]{exam}
\RequirePackage{amssymb, amsfonts, amsmath, latexsym, verbatim, xspace, setspace, mathrsfs}
\usepackage{amsmath,amsthm,amssymb,amsfonts, 
hyperref, color, graphicx, enumitem}
\RequirePackage{tikz, pgflibraryplotmarks}
\usepackage{geometry}
\usepackage{graphicx}

\newcommand{\N}{\mathbb{N}}
\newcommand{\Z}{\mathbb{Z}}
\newcommand{\Q}{\mathbb{Q}}
 \everymath{\displaystyle}
\newenvironment{problem}[2][Problem:]{\begin{trivlist}
\item[\hskip \labelsep {\bfseries #1}\hskip \labelsep {\bfseries #2}]}{\end{trivlist}}

\newenvironment{claim}[2][Claim:]{\begin{trivlist}
\item[\hskip \labelsep {\bfseries #1}\hskip \labelsep {\bfseries #2}]}{\end{trivlist}}

\newenvironment{defn}[2][Definition:]{\begin{trivlist}
\item[\hskip \labelsep {\bfseries #1}\hskip \labelsep {\bfseries #2}]}{\end{trivlist}}

% Here's where you edit the Class, Exam, Date, etc.
\newcommand{\class}{Sec 4}
\newcommand{\term}{Winter 2019}
\newcommand{\examnum}{}
\newcommand{\examdate}{Exam 2}
\newcommand{\timelimit}{90 min}

% For an exam, single spacing is most appropriate
\singlespacing
% \onehalfspacing
% \doublespacing

% For an exam, we generally want to turn off paragraph indentation
\parindent 0ex

\begin{document} 

% These commands set up the running header on the top of the exam pages
\pagestyle{head}
\firstpageheader{Name:}{Exam 2}{}
\runningheader{\class}{\examnum\ - Page \thepage\ of \numpages}{\examdate}
\runningheadrule

\begin{flushright}
\begin{tabular}{p{2.8in} r l}
\textbf{\class} & \textbf{Name:} & \makebox[2in]{\hrulefill}\\
\textbf{\term} &&\textbf{\examnum}\\
\textbf{\examdate} &&
\textbf{Time Limit: \timelimit}  \\ 
\end{tabular}\\
\end{flushright}
\rule[1ex]{\textwidth}{.1pt}




\begin{minipage}[t]{3.7in}
\vspace{0pt}
\begin{itemize}

\item \textbf{DO NOT open the exam booklet until you are told to begin. You should write your name and section number at the top and read the instructions.}

\vfill

\item Organize your work, in a reasonably neat and coherent way, in
the space provided. If you wish for something to not be graded, please strike it out neatly. I will grade only work on the exam paper, unless you clearly indicate your desire for me to grade work on additional pages.

\item You may use any results from class, homework or the text, but you must cite the result you are using. You must prove everything else.

\item You needn't spend your time rewriting definitions or axioms on the exam.

\end{itemize}


\end{minipage}
\hfill
\begin{minipage}[t]{2.3in}
\vspace{0pt}
%\cellwidth{3em}
\gradetablestretch{2}
%Uncomment this line to make the table display 100 as the total no matter what. This is good for tests with an ommit question.
%\settabletotalpoints{100}
\vqword{Problem}
\addpoints % required here by exam.cls, even though questions haven't started yet.	
\gradetable[v]%[pages]  % Use [pages] to have grading table by page instead of question

\end{minipage}

\begin{itemize}

\item When you have completed your test, hand it to me and go have a great weekend!


\end{itemize}

\newpage

\begin{questions}
\addpoints
\question[5] 
The sum of three consecutive even numbers is 246.  What are the integers?
    \vfill

\question[5] A girl is 5 years younger than her sister and the product of their ages is 204.  How old is the older sister?
\vfill

\newpage 
\addpoints
\question[10] Solve and check your result.
$$
 \displaystyle \left\{\begin{aligned}[lcr] 2x-6y & = &  -24 \\ x-5y & = &  -22 \end{aligned}\right.
$$
\vfill
\question[10] If 100 coins consists of quarters and dimes, how many of each are there if they are \$20.80 in total?
\vfill

\newpage 
\addpoints

\question
\begin{parts}Solve,

  \part[5] $4(x+3)^2 - 3 = 17$
  \vfill

  \part[5] $n^2+8n+15 = 0$
  \vfill

  \part[5] $x^2-x+1=0$
  \vfill
  
\end{parts}



\newpage
\addpoints
\question Find the solutions of each expression.
\begin{parts}
  \part[5] $\displaystyle \frac{4}{y-4} - \frac{3}{y-3} = 1$
  \vfill
  \part[5] $-3+2\sqrt{x+1} = 7$
\vfill
\end{parts}

%% \question Simplify and radicalize the denominator.
%% \begin{parts}
%%   \part[3] $\frac{1}{\sqrt{7}}$
%%   \vfill
%%   \part[3] $\frac{\sqrt{11}+\sqrt{5}}{\sqrt{11}-\sqrt{5}}$
%%   \vfill
%% \end{parts}


%\bonusquestion[5] This is a bonus question. It has points but they are not added on the cover page.
%\vfill

\end{questions}
\end{document}
